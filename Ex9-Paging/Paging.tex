\documentclass[12pt,letterpaper,,margin=0.75in]{article}
\usepackage[utf8]{inputenc}
\usepackage[english]{babel}

\usepackage{listings}
\usepackage{xcolor}
%\usepackage{minted}

%For syntax highlighting
\definecolor{codegreen}{rgb}{0,0.6,0}
\definecolor{codegray}{rgb}{0.5,0.5,0.5}
\definecolor{codepurple}{rgb}{0.58,0,0.82}
\definecolor{backcolour}{rgb}{1,1,1}

%%Sets different parameters
\lstdefinestyle{mystyle}{
	backgroundcolor=\color{backcolour},   
    commentstyle=\color{codegreen},
    keywordstyle=\color{magenta},
    numberstyle=\tiny\color{codegray},
    stringstyle=\color{codepurple},
    basicstyle=\ttfamily\footnotesize,
    breakatwhitespace=false,         
    breaklines=true,                 
    captionpos=b,                    
    keepspaces=true,                 
    numbers=left,                    
    numbersep=5pt,                  
    showspaces=false,                
    showstringspaces=false,
    showtabs=false,                  
    tabsize=4
}
\lstset{style=mystyle}

%title
\title{\textbf{Department of Computer Science and Engineering}}
\author{\textbf{S.G.Shivanirudh , 185001146, Semester IV }}
\date{26 March 2020}

\begin{document}
\maketitle
\hrule
\section*{\center{UCS1411 - Operating Systems Laboratory}}
\hrule 
\bigskip \bigskip

%Assignment name
\subsection*{\center{\textbf{Lab Exercise 9: Implementation of Paging Technique}}}

%Objective
\subsection*{\flushleft{\emph{Objective:}}}
\begin{flushleft}
Develop a C program to implement the paging technique in memory management.
\end{flushleft}

%Code
\subsection*{\flushleft{\emph{Code:}}}

\flushleft{Q.To write a C program to implement the paging technique in memory management.}
\lstinputlisting[language=C]{Queue.h}
\lstinputlisting[language=C,firstline=2,lastline=132]{Paging.c}
%Output
\flushleft{\textbf{\emph{Output:}}}
\lstinputlisting[language=C,firstline=137,lastline=273]{Paging.c}
\hrule
\vspace{10mm}

\end{document}